\documentclass[11pt]{article}\usepackage[]{graphicx}\usepackage[]{xcolor}
% maxwidth is the original width if it is less than linewidth
% otherwise use linewidth (to make sure the graphics do not exceed the margin)
\makeatletter
\def\maxwidth{ %
  \ifdim\Gin@nat@width>\linewidth
    \linewidth
  \else
    \Gin@nat@width
  \fi
}
\makeatother

\setcounter{secnumdepth}{0} % No numbering for sections and below

\definecolor{fgcolor}{rgb}{0.345, 0.345, 0.345}
\newcommand{\hlnum}[1]{\textcolor[rgb]{0.686,0.059,0.569}{#1}}%
\newcommand{\hlsng}[1]{\textcolor[rgb]{0.192,0.494,0.8}{#1}}%
\newcommand{\hlcom}[1]{\textcolor[rgb]{0.678,0.584,0.686}{\textit{#1}}}%
\newcommand{\hlopt}[1]{\textcolor[rgb]{0,0,0}{#1}}%
\newcommand{\hldef}[1]{\textcolor[rgb]{0.345,0.345,0.345}{#1}}%
\newcommand{\hlkwa}[1]{\textcolor[rgb]{0.161,0.373,0.58}{\textbf{#1}}}%
\newcommand{\hlkwb}[1]{\textcolor[rgb]{0.69,0.353,0.396}{#1}}%
\newcommand{\hlkwc}[1]{\textcolor[rgb]{0.333,0.667,0.333}{#1}}%
\newcommand{\hlkwd}[1]{\textcolor[rgb]{0.737,0.353,0.396}{\textbf{#1}}}%
\newcommand{\Rlogo}{\protect\includegraphics[height=1.8ex,keepaspectratio]{Rlogo.png}}
\let\hlipl\hlkwb

\usepackage{framed}
\makeatletter
\newenvironment{kframe}{%
 \def\at@end@of@kframe{}%
 \ifinner\ifhmode%
  \def\at@end@of@kframe{\end{minipage}}%
  \begin{minipage}{\columnwidth}%
 \fi\fi%
 \def\FrameCommand##1{\hskip\@totalleftmargin \hskip-\fboxsep
 \colorbox{shadecolor}{##1}\hskip-\fboxsep
     % There is no \\@totalrightmargin, so:
     \hskip-\linewidth \hskip-\@totalleftmargin \hskip\columnwidth}%
 \MakeFramed {\advance\hsize-\width
   \@totalleftmargin\z@ \linewidth\hsize
   \@setminipage}}%
 {\par\unskip\endMakeFramed%
 \at@end@of@kframe}
\makeatother

\definecolor{shadecolor}{rgb}{.97, .97, .97}
\definecolor{messagecolor}{rgb}{0, 0, 0}
\definecolor{warningcolor}{rgb}{1, 0, 1}
\definecolor{errorcolor}{rgb}{1, 0, 0}
\newenvironment{knitrout}{}{} % an empty environment to be redefined in TeX

\usepackage{alltt}
\usepackage{hanging}
\usepackage{longtable}

% \renewcommand\thefigure{L-\arabic{figure}}
% \renewcommand\thetable{L-\arabic{table}}
% \renewcommand\thepage{L-\arabic{page}}
% \renewcommand\theequation{L\arabic{equation}}

\newcommand\vsp{\vspace{2mm}}

\usepackage{fullpage}
\usepackage{bm}

\usepackage{amsmath,amsthm}
\usepackage{nameref,hyperref}
\usepackage[normalem]{ulem}% to use \sout in feedback commands
\usepackage{comment}
\usepackage{enumitem}
% \usepackage{wallpaper}

\excludecomment{hidden}
% \bibliographystyle{apalike}

% Use the PLoS provided BiBTeX style
% \bibliographystyle{../plos2015}
% \usepackage{cite}

% Remove brackets from numbering in List of References
\makeatletter
\renewcommand{\@biblabel}[1]{\quad#1.}
\makeatother

\usepackage{verbatim}
\usepackage{graphicx}
\usepackage{amsmath,amssymb}
\usepackage{amsfonts}
 \usepackage{url}
\usepackage{color}

\newcommand\code[1]{\texttt{#1}}
\newcommand\paramVec{\theta}

\newcommand\seq[2]{{#1}\!:\!{#2}}
\newcommand\vaccClass{Z}
\newcommand\vaccCounter{z}
\newcommand\muRS{\mu_{RS}}
\newcommand\sigmaProc{\sigma_{\mathrm{proc}}}
\newcommand\transmissionTrend{\zeta}
\newcommand\transmission{\beta}
% \input{../edits.tex}
%\usepackage[sectionbib]{natbib}

%\usepackage{xr}
%\externaldocument{../ms}

%\newcommand\ifLetter[2]{{#1}}

\parskip 7pt
\parindent 0pt
\setlength\parindent{0pt}
\newcommand\report[1]{{\color{mygreen} \vspace{1mm}\hspace{0.25in}\parbox{6in}{\em #1}}}
\newcommand\reportN[1]{{\color{mygreen} \vspace{1mm}\hspace{0.35in}\parbox{5.75in}{\em #1}}}
\newcommand\article[1]{{\color{blue} \vspace{1mm}\hspace{0.25in}\parbox{6in}{\em #1}}}
\newcommand\blue[1]{{\color{blue}{#1}}}

%\topmargin -0.3in % for EI dvips
\IfFileExists{upquote.sty}{\usepackage{upquote}}{}
\begin{document}

% \ThisULCornerWallPaper{1}{ISUletterhead2025.pdf}



\rule{0cm}{0.3cm}

%\vspace{-28mm}
\vspace{-8mm}

\includegraphics[width=\textwidth,trim={0 25.2cm 0 1.6cm},clip]{ISUletterhead2025.pdf}

% \vspace{10mm}

\begin{center}
{\bf \LARGE Course Overview and Syllabus: Math 3350, Fall 2025}
\end{center}

\vspace{3mm}

Course Website: \href{https://jeswheel.github.io/3350\_f25}{https://jeswheel.github.io/3350\_f25}

\section{Instructor Information}

Jesse Wheeler \\
Mathematics and Statistics Department \\
Email: jessewheeler@isu.edu\\
Office: PS 314C\\
Office Hours: 
\begin{itemize}
  \item TBD
\end{itemize}

You can contact me during office hours or via Email. 
I will try my best to respond to questions within 24 hours during Mon-Fri.
Feel free to email me over the weekends, but I may not be actively monitoring my emails during these times.

\section{Course Description}

A calculus-based introduction to statistical procedures, including simple regression, basic experimental design, and non-parametric methods. 
PREREQ: MATH 1160 or MATH 1170.

\section{Course Format}

This is a standard class where we will meet in-person.
Course materials, activities, and assignments will be posted on Canvas.\\ 
\textbf{General course information can be found at the course website:} \url{https://jeswheel.github.io/3350\_f25}

\section{Textbook and Course Materials}

\begin{hangparas}{0.25in}{1}
Baldi, Brigitte, and David S. Moore. \emph{The practice of statistics in the life sciences}. Freeman, 2012. ISBN: 9781319403348.
\end{hangparas}

\subsection{Software}

This semester we will use the \Rlogo\, programming language; 
for the purpose of our class, think of this software as a special calculator used for Statistics.
To interact with this software, I highly recommend using RStudio, an Integrated Development Environment (IDE) built for writing \Rlogo\, code.

\textbf{Why R?} R and RStudio are both free, and widely used software in both academia and industry. 
R has been built specifically to help practicioners do statistics easily.

\begin{itemize}
  \item Installing \Rlogo: \href{https://cran.r-project.org/}{https://cran.r-project.org/}
  \item Installing RStudio: \href{https://posit.co/download/rstudio-desktop/}{https://posit.co/download/rstudio-desktop/}. 
  Note that you should install \Rlogo\, prior to installing RStudio.
\end{itemize}

\section{Grading}

\begin{table}[h!]
\centering
\begin{tabular}{|c|c|}
\hline
\textbf{Grade} & \textbf{Percentage Range} \\
\hline
Participation Quizes & $5\%$ \\
Homework & $50\%$ \\
Midterm & $15\%$ \\
Final & $30\%$ \\
\hline
\end{tabular}
\caption{Assignment Weights}
\end{table}

\begin{table}[h!]
\centering
\begin{tabular}{|c|c|}
\hline
\textbf{Grade} & \textbf{Percentage Range} \\
\hline
A  & 93--100 \\
A- & 90--92.99 \\\hline
B+ & 87--89.99 \\
B  & 83--86.99 \\
B- & 80--82.99 \\\hline
C+ & 77--79.99 \\
C  & 73--76.99 \\
C- & 70--72.99 \\\hline
D+ & 67--69.99 \\
D  & 65--66.99 \\
D- & 60--64.99 \\\hline
F  & 0--54.99 \\
\hline
\end{tabular}
\caption{Grade Breakdown}
\end{table}

\section{Schedule (Tentative)}

The table below provides a tentative schedule of the course. 
This table will be updated on the course website as needed.

\begin{longtable}{|c|l|l|l|}
\hline
\textbf{Week} & \textbf{Dates (MWF)} & \textbf{Textbook Chapters} & \textbf{Topic} \\
\hline
\endhead

1 & Aug 25, 27, 29 & Chapter 1, 2 & Graphs and Summaries \\ \hline
2 & Sep {\color{red} \sout{1}}, 3, 5 & Chapter 3 & Scatterplots and Correlation \\ \hline
3 & Sep 8, 10, 12 & Chapter 4 & Regression \\ \hline
4 & Sep 15, 17, 19 & Chapter 5 & Two-Way Tables \\ \hline
5 & Sep 22, 24, 26 & Chapter 6, 7 & Samples vs Experiments \\ \hline
6 & Sep 29, Oct 1, 3 & Chapter 9 & Probability \\ \hline
7 & Oct 6, 8, 10 & Chapter 10 & Independence \\ \hline
8 & Oct 13, 15, {\color{blue} \textbf{17}} & Chapter 11, 12 & Common Distributions \\ \hline
9 & Oct 20, 22, 24 & Chapter 13 & Sampling Distributions \\ \hline
10 & Oct 27, 29, 31 & Chapter 14 & Inference \\ \hline
11 & Nov 3, 5, 7 & Chapter 15 & Inference in Practice\\ \hline
12 & Nov 10, 12, 14 & Chapter 17-18 & Inference for Population Mean \\ \hline
13 & Nov 17, 19, 21 & Chapter 18-19 & Inference for Proportions\\ \hline
14 & Nov 24, 26, 28 & \multicolumn{2}{l|}{Fall Break - No Classes} \\ \hline
15 & Dec 1, 3, 5 & Chapter 20 & Comparing Proportions \\ \hline
16 & Dec 8, 10, 12 & Chapters 21-23 & Additional Topics \\ \hline
17 & Dec 15--19 & \multicolumn{2}{l|}{Finals Week (Dec 17)} \\ \hline

\end{longtable}

For other important dates and deadlines, please see the University academic calendar: \url{https://www.isu.edu/academiccalendar/}.

\subsection{Exams}

\begin{itemize}
  \item \textbf{Final Exam}: Wednesday, Dec 17, 7:30-9:30 a.m. (Sorry about the time, I didn't chose this). The location will be our regular classroom. 
  \item \textbf{Midterm}: Planned for {\color{blue} \textbf{Oct 17}}, during regular class time. Tentatively, the midterm will cover Chapters 1-10.
\end{itemize}

\end{document}

